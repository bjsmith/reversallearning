\documentclass[]{article}
\usepackage{lmodern}
\usepackage{amssymb,amsmath}
\usepackage{ifxetex,ifluatex}
\usepackage{fixltx2e} % provides \textsubscript
\ifnum 0\ifxetex 1\fi\ifluatex 1\fi=0 % if pdftex
  \usepackage[T1]{fontenc}
  \usepackage[utf8]{inputenc}
\else % if luatex or xelatex
  \ifxetex
    \usepackage{mathspec}
  \else
    \usepackage{fontspec}
  \fi
  \defaultfontfeatures{Ligatures=TeX,Scale=MatchLowercase}
\fi
% use upquote if available, for straight quotes in verbatim environments
\IfFileExists{upquote.sty}{\usepackage{upquote}}{}
% use microtype if available
\IfFileExists{microtype.sty}{%
\usepackage{microtype}
\UseMicrotypeSet[protrusion]{basicmath} % disable protrusion for tt fonts
}{}
\usepackage[margin=1in]{geometry}
\usepackage{hyperref}
\hypersetup{unicode=true,
            pdftitle={Comparing weak and non-informative priors},
            pdfauthor={Ben Smith},
            pdfborder={0 0 0},
            breaklinks=true}
\urlstyle{same}  % don't use monospace font for urls
\usepackage{longtable,booktabs}
\usepackage{graphicx,grffile}
\makeatletter
\def\maxwidth{\ifdim\Gin@nat@width>\linewidth\linewidth\else\Gin@nat@width\fi}
\def\maxheight{\ifdim\Gin@nat@height>\textheight\textheight\else\Gin@nat@height\fi}
\makeatother
% Scale images if necessary, so that they will not overflow the page
% margins by default, and it is still possible to overwrite the defaults
% using explicit options in \includegraphics[width, height, ...]{}
\setkeys{Gin}{width=\maxwidth,height=\maxheight,keepaspectratio}
\IfFileExists{parskip.sty}{%
\usepackage{parskip}
}{% else
\setlength{\parindent}{0pt}
\setlength{\parskip}{6pt plus 2pt minus 1pt}
}
\setlength{\emergencystretch}{3em}  % prevent overfull lines
\providecommand{\tightlist}{%
  \setlength{\itemsep}{0pt}\setlength{\parskip}{0pt}}
\setcounter{secnumdepth}{0}
% Redefines (sub)paragraphs to behave more like sections
\ifx\paragraph\undefined\else
\let\oldparagraph\paragraph
\renewcommand{\paragraph}[1]{\oldparagraph{#1}\mbox{}}
\fi
\ifx\subparagraph\undefined\else
\let\oldsubparagraph\subparagraph
\renewcommand{\subparagraph}[1]{\oldsubparagraph{#1}\mbox{}}
\fi

%%% Use protect on footnotes to avoid problems with footnotes in titles
\let\rmarkdownfootnote\footnote%
\def\footnote{\protect\rmarkdownfootnote}

%%% Change title format to be more compact
\usepackage{titling}

% Create subtitle command for use in maketitle
\newcommand{\subtitle}[1]{
  \posttitle{
    \begin{center}\large#1\end{center}
    }
}

\setlength{\droptitle}{-2em}
  \title{Comparing weak and non-informative priors}
  \pretitle{\vspace{\droptitle}\centering\huge}
  \posttitle{\par}
  \author{Ben Smith}
  \preauthor{\centering\large\emph}
  \postauthor{\par}
  \predate{\centering\large\emph}
  \postdate{\par}
  \date{6/4/2018}


\begin{document}
\maketitle

\begin{verbatim}
##  [1]  1  2  3  4  5  6  7  8  9 10
\end{verbatim}

I compared models with 500 iterations, sampled from 450 to 500, with a
delta adaption of 0.9. Both models were identical, and used non-centered
priors, except that one model used weak, uninformative priors and one
model used informative priors.

I compared the first 10 subjects in the dataset.

Priors were:

\begin{longtable}[]{@{}lrr@{}}
\toprule
& Weak Priors & Informative Priors\tabularnewline
\midrule
\endhead
subject \(\alpha_{\mu\mu}\) & -3.00 & -3.24\tabularnewline
subject \(\alpha_{\mu\sigma}\) & 3.00 & 1.89\tabularnewline
subject \(\alpha_{\sigma}\) & 5.00 & 2.07\tabularnewline
run \(\alpha_{\sigma_{\gamma}}\) & 4.00 & 2.07\tabularnewline
subject \(k_{\mu\mu}\) & -0.69 & -0.31\tabularnewline
subject \(k_{\mu\sigma}\) & 1.00 & 0.12\tabularnewline
subject \(k_{\sigma}\) & 3.00 & 0.51\tabularnewline
run \(k_{\sigma_{\gamma}}\) & 2.00 & 0.35\tabularnewline
subject \(\tau_{\mu\mu}\) & -0.69 & -1.75\tabularnewline
subject \(\tau_{\mu\sigma}\) & 0.50 & 0.18\tabularnewline
subject \(\tau_{\sigma}\) & 2.00 & 1.23\tabularnewline
run \(\tau_{\sigma_{\gamma}}\) & 1.00 & 0.97\tabularnewline
\bottomrule
\end{longtable}

These priors were defined in the manner described in
ref\{sec:CalculatingEmpiricalPriors\}.

How did they do?

\subsection{Effiency}\label{effiency}

I wanted to measure efficiency for each parameter. At maximum
efficiency, the effective sample size per chain-iteration is 1; lower
efficiencies are degraditions of this.

\begin{longtable}[]{@{}lrr@{}}
\caption{Efficiency per chain-iteration by parameter}\tabularnewline
\toprule
& Weak priors & Informative priors\tabularnewline
\midrule
\endfirsthead
\toprule
& Weak priors & Informative priors\tabularnewline
\midrule
\endhead
subject \(\alpha_{\mu}\) & 0.12 & 0.43\tabularnewline
subject \(k_{\mu}\) & 0.67 & 0.49\tabularnewline
subject \(\tau_{\mu}\) & 0.17 & 0.70\tabularnewline
subject \(\alpha_{\sigma}\) & 0.87 & 0.48\tabularnewline
subject \(k_{\sigma}\) & 0.51 & 0.53\tabularnewline
subject \(\tau_{\sigma}\) & 0.14 & 0.34\tabularnewline
run \(\alpha_{\sigma_{\gamma}}\) & 1.00 & 1.00\tabularnewline
run \(k_{\sigma_{\gamma}}\) & 1.00 & 1.00\tabularnewline
run \(\tau_{\sigma_{\gamma}}\) & 1.00 & 1.00\tabularnewline
\bottomrule
\end{longtable}

For estimating group parameters, efficiency seemed better overall for
the informative prior model, although there's no clear winner.

\subsection{Representativeness}\label{representativeness}

Representativeness is measured by the Gelman-Rubin \(\hat{R}\)
statistic, and should ideally be lower than 1.05.

\begin{longtable}[]{@{}lrr@{}}
\toprule
& WeakPriors & InformativePriors\tabularnewline
\midrule
\endhead
subject \(\alpha_{\mu}\) & 1.07 & 1.05\tabularnewline
subject \(k_{\mu}\) & 0.99 & 1.01\tabularnewline
subject \(\tau_{\mu}\) & 1.09 & 1.02\tabularnewline
subject \(\alpha_{\sigma}\) & 1.00 & 1.02\tabularnewline
subject \(k_{\sigma}\) & 1.02 & 1.01\tabularnewline
subject \(\tau_{\sigma}\) & 1.12 & 1.00\tabularnewline
run \(\alpha_{\sigma_{\gamma}}\) & 0.99 & 1.00\tabularnewline
run \(k_{\sigma_{\gamma}}\) & 1.00 & 0.99\tabularnewline
run \(\tau_{\sigma_{\gamma}}\) & 1.01 & 1.02\tabularnewline
\bottomrule
\end{longtable}

I did seem to find evidence that the informative priors allowed for a
better fit; these conclusions should be held tentatively until we can
run the test several times repeatedly with varying starting seeds.
Still, while using weak priors, two of the three group mean values had
\(\hat{R}\) values above 1.05, whereas for informative priors, all
values were below 1.05.


\end{document}
