\documentclass[]{article}
\usepackage{lmodern}
\usepackage{amssymb,amsmath}
\usepackage{ifxetex,ifluatex}
\usepackage{fixltx2e} % provides \textsubscript
\ifnum 0\ifxetex 1\fi\ifluatex 1\fi=0 % if pdftex
  \usepackage[T1]{fontenc}
  \usepackage[utf8]{inputenc}
\else % if luatex or xelatex
  \ifxetex
    \usepackage{mathspec}
  \else
    \usepackage{fontspec}
  \fi
  \defaultfontfeatures{Ligatures=TeX,Scale=MatchLowercase}
\fi
% use upquote if available, for straight quotes in verbatim environments
\IfFileExists{upquote.sty}{\usepackage{upquote}}{}
% use microtype if available
\IfFileExists{microtype.sty}{%
\usepackage{microtype}
\UseMicrotypeSet[protrusion]{basicmath} % disable protrusion for tt fonts
}{}
\usepackage[margin=1in]{geometry}
\usepackage{hyperref}
\hypersetup{unicode=true,
            pdftitle={All about reaction times},
            pdfauthor={Ben Smith},
            pdfborder={0 0 0},
            breaklinks=true}
\urlstyle{same}  % don't use monospace font for urls
\usepackage{graphicx,grffile}
\makeatletter
\def\maxwidth{\ifdim\Gin@nat@width>\linewidth\linewidth\else\Gin@nat@width\fi}
\def\maxheight{\ifdim\Gin@nat@height>\textheight\textheight\else\Gin@nat@height\fi}
\makeatother
% Scale images if necessary, so that they will not overflow the page
% margins by default, and it is still possible to overwrite the defaults
% using explicit options in \includegraphics[width, height, ...]{}
\setkeys{Gin}{width=\maxwidth,height=\maxheight,keepaspectratio}
\IfFileExists{parskip.sty}{%
\usepackage{parskip}
}{% else
\setlength{\parindent}{0pt}
\setlength{\parskip}{6pt plus 2pt minus 1pt}
}
\setlength{\emergencystretch}{3em}  % prevent overfull lines
\providecommand{\tightlist}{%
  \setlength{\itemsep}{0pt}\setlength{\parskip}{0pt}}
\setcounter{secnumdepth}{0}
% Redefines (sub)paragraphs to behave more like sections
\ifx\paragraph\undefined\else
\let\oldparagraph\paragraph
\renewcommand{\paragraph}[1]{\oldparagraph{#1}\mbox{}}
\fi
\ifx\subparagraph\undefined\else
\let\oldsubparagraph\subparagraph
\renewcommand{\subparagraph}[1]{\oldsubparagraph{#1}\mbox{}}
\fi

%%% Use protect on footnotes to avoid problems with footnotes in titles
\let\rmarkdownfootnote\footnote%
\def\footnote{\protect\rmarkdownfootnote}

%%% Change title format to be more compact
\usepackage{titling}

% Create subtitle command for use in maketitle
\newcommand{\subtitle}[1]{
  \posttitle{
    \begin{center}\large#1\end{center}
    }
}

\setlength{\droptitle}{-2em}
  \title{All about reaction times}
  \pretitle{\vspace{\droptitle}\centering\huge}
  \posttitle{\par}
  \author{Ben Smith}
  \preauthor{\centering\large\emph}
  \postauthor{\par}
  \predate{\centering\large\emph}
  \postdate{\par}
  \date{6/4/2018}


\begin{document}
\maketitle

\begin{verbatim}
## -------------------------------------------------------------------------
\end{verbatim}

\begin{verbatim}
## You have loaded plyr after dplyr - this is likely to cause problems.
## If you need functions from both plyr and dplyr, please load plyr first, then dplyr:
## library(plyr); library(dplyr)
\end{verbatim}

\begin{verbatim}
## -------------------------------------------------------------------------
\end{verbatim}

\begin{verbatim}
## 
## Attaching package: 'plyr'
\end{verbatim}

\begin{verbatim}
## The following objects are masked from 'package:dplyr':
## 
##     arrange, count, desc, failwith, id, mutate, rename, summarise,
##     summarize
\end{verbatim}

\begin{verbatim}
## Loading required package: R.matlab
\end{verbatim}

\begin{verbatim}
## R.matlab v3.6.1 (2016-10-19) successfully loaded. See ?R.matlab for help.
\end{verbatim}

\begin{verbatim}
## 
## Attaching package: 'R.matlab'
\end{verbatim}

\begin{verbatim}
## The following objects are masked from 'package:base':
## 
##     getOption, isOpen
\end{verbatim}

\begin{verbatim}
## Loading required package: tidyr
\end{verbatim}

\begin{verbatim}
## 
## Attaching package: 'tidyr'
\end{verbatim}

\begin{verbatim}
## The following object is masked from 'package:rstan':
## 
##     extract
\end{verbatim}

\begin{verbatim}
## Loading required package: corrplot
\end{verbatim}

\begin{verbatim}
## corrplot 0.84 loaded
\end{verbatim}

\section{Exclusion of abnormally short reaction
times}\label{exclusion-of-abnormally-short-reaction-times}

It was found that in the linear ballistic accumulator model, extremely
short reaction times can result in an infinite likelihood if they are
treated as a genuine response to the stimulus. Because according the
model, these response times have zero likelihood, assuming that overall
model parameters are approximately correct, they must be something other
than an actual response, e.g., a simple accidental button press that
coincides with the presentation of a stimulus.

The distribution of response times is bimodal:

\begin{verbatim}
## Warning: Transformation introduced infinite values in continuous y-axis
\end{verbatim}

\begin{verbatim}
## Warning: Removed 27 rows containing missing values (geom_bar).
\end{verbatim}

\includegraphics{reaction_time_files/figure-latex/ResponseTimeDistribution-1.pdf}
Due to a technical error, there were 7 trials where reaction times
between 1.1 and 1.5 were recorded. These were not excluded as they were
likely due to a technical error in code or temporary latency problem
rather than an erroneous response.

In a simple reaction time task, textcite\{woods2015factors\} found
subjects' simple response time had a mean of 213 ms, including movement
initiation. In their analysis, they coded responses less than
110\textasciitilde{}ms as a false alarm. However, considering the
distribution observed in this study, along with the relatively high cost
of including a false response compared to teh relatively negative cost
of excluding at true response, I used a slightly higher threshold of
140\textasciitilde{}ms.

\section{Excluding subjects who were performing abnormally
poorly}\label{excluding-subjects-who-were-performing-abnormally-poorly}

NEED TO WRITE THIS IN AS WELL.


\end{document}
