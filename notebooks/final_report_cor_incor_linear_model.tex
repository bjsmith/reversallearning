\documentclass[]{article}
\usepackage{lmodern}
\usepackage{amssymb,amsmath}
\usepackage{ifxetex,ifluatex}
\usepackage{fixltx2e} % provides \textsubscript
\ifnum 0\ifxetex 1\fi\ifluatex 1\fi=0 % if pdftex
  \usepackage[T1]{fontenc}
  \usepackage[utf8]{inputenc}
\else % if luatex or xelatex
  \ifxetex
    \usepackage{mathspec}
  \else
    \usepackage{fontspec}
  \fi
  \defaultfontfeatures{Ligatures=TeX,Scale=MatchLowercase}
\fi
% use upquote if available, for straight quotes in verbatim environments
\IfFileExists{upquote.sty}{\usepackage{upquote}}{}
% use microtype if available
\IfFileExists{microtype.sty}{%
\usepackage{microtype}
\UseMicrotypeSet[protrusion]{basicmath} % disable protrusion for tt fonts
}{}
\usepackage[margin=1in]{geometry}
\usepackage{hyperref}
\hypersetup{unicode=true,
            pdftitle={R Notebook},
            pdfborder={0 0 0},
            breaklinks=true}
\urlstyle{same}  % don't use monospace font for urls
\usepackage{color}
\usepackage{fancyvrb}
\newcommand{\VerbBar}{|}
\newcommand{\VERB}{\Verb[commandchars=\\\{\}]}
\DefineVerbatimEnvironment{Highlighting}{Verbatim}{commandchars=\\\{\}}
% Add ',fontsize=\small' for more characters per line
\usepackage{framed}
\definecolor{shadecolor}{RGB}{248,248,248}
\newenvironment{Shaded}{\begin{snugshade}}{\end{snugshade}}
\newcommand{\KeywordTok}[1]{\textcolor[rgb]{0.13,0.29,0.53}{\textbf{{#1}}}}
\newcommand{\DataTypeTok}[1]{\textcolor[rgb]{0.13,0.29,0.53}{{#1}}}
\newcommand{\DecValTok}[1]{\textcolor[rgb]{0.00,0.00,0.81}{{#1}}}
\newcommand{\BaseNTok}[1]{\textcolor[rgb]{0.00,0.00,0.81}{{#1}}}
\newcommand{\FloatTok}[1]{\textcolor[rgb]{0.00,0.00,0.81}{{#1}}}
\newcommand{\ConstantTok}[1]{\textcolor[rgb]{0.00,0.00,0.00}{{#1}}}
\newcommand{\CharTok}[1]{\textcolor[rgb]{0.31,0.60,0.02}{{#1}}}
\newcommand{\SpecialCharTok}[1]{\textcolor[rgb]{0.00,0.00,0.00}{{#1}}}
\newcommand{\StringTok}[1]{\textcolor[rgb]{0.31,0.60,0.02}{{#1}}}
\newcommand{\VerbatimStringTok}[1]{\textcolor[rgb]{0.31,0.60,0.02}{{#1}}}
\newcommand{\SpecialStringTok}[1]{\textcolor[rgb]{0.31,0.60,0.02}{{#1}}}
\newcommand{\ImportTok}[1]{{#1}}
\newcommand{\CommentTok}[1]{\textcolor[rgb]{0.56,0.35,0.01}{\textit{{#1}}}}
\newcommand{\DocumentationTok}[1]{\textcolor[rgb]{0.56,0.35,0.01}{\textbf{\textit{{#1}}}}}
\newcommand{\AnnotationTok}[1]{\textcolor[rgb]{0.56,0.35,0.01}{\textbf{\textit{{#1}}}}}
\newcommand{\CommentVarTok}[1]{\textcolor[rgb]{0.56,0.35,0.01}{\textbf{\textit{{#1}}}}}
\newcommand{\OtherTok}[1]{\textcolor[rgb]{0.56,0.35,0.01}{{#1}}}
\newcommand{\FunctionTok}[1]{\textcolor[rgb]{0.00,0.00,0.00}{{#1}}}
\newcommand{\VariableTok}[1]{\textcolor[rgb]{0.00,0.00,0.00}{{#1}}}
\newcommand{\ControlFlowTok}[1]{\textcolor[rgb]{0.13,0.29,0.53}{\textbf{{#1}}}}
\newcommand{\OperatorTok}[1]{\textcolor[rgb]{0.81,0.36,0.00}{\textbf{{#1}}}}
\newcommand{\BuiltInTok}[1]{{#1}}
\newcommand{\ExtensionTok}[1]{{#1}}
\newcommand{\PreprocessorTok}[1]{\textcolor[rgb]{0.56,0.35,0.01}{\textit{{#1}}}}
\newcommand{\AttributeTok}[1]{\textcolor[rgb]{0.77,0.63,0.00}{{#1}}}
\newcommand{\RegionMarkerTok}[1]{{#1}}
\newcommand{\InformationTok}[1]{\textcolor[rgb]{0.56,0.35,0.01}{\textbf{\textit{{#1}}}}}
\newcommand{\WarningTok}[1]{\textcolor[rgb]{0.56,0.35,0.01}{\textbf{\textit{{#1}}}}}
\newcommand{\AlertTok}[1]{\textcolor[rgb]{0.94,0.16,0.16}{{#1}}}
\newcommand{\ErrorTok}[1]{\textcolor[rgb]{0.64,0.00,0.00}{\textbf{{#1}}}}
\newcommand{\NormalTok}[1]{{#1}}
\usepackage{longtable,booktabs}
\usepackage{graphicx,grffile}
\makeatletter
\def\maxwidth{\ifdim\Gin@nat@width>\linewidth\linewidth\else\Gin@nat@width\fi}
\def\maxheight{\ifdim\Gin@nat@height>\textheight\textheight\else\Gin@nat@height\fi}
\makeatother
% Scale images if necessary, so that they will not overflow the page
% margins by default, and it is still possible to overwrite the defaults
% using explicit options in \includegraphics[width, height, ...]{}
\setkeys{Gin}{width=\maxwidth,height=\maxheight,keepaspectratio}
\IfFileExists{parskip.sty}{%
\usepackage{parskip}
}{% else
\setlength{\parindent}{0pt}
\setlength{\parskip}{6pt plus 2pt minus 1pt}
}
\setlength{\emergencystretch}{3em}  % prevent overfull lines
\providecommand{\tightlist}{%
  \setlength{\itemsep}{0pt}\setlength{\parskip}{0pt}}
\setcounter{secnumdepth}{0}
% Redefines (sub)paragraphs to behave more like sections
\ifx\paragraph\undefined\else
\let\oldparagraph\paragraph
\renewcommand{\paragraph}[1]{\oldparagraph{#1}\mbox{}}
\fi
\ifx\subparagraph\undefined\else
\let\oldsubparagraph\subparagraph
\renewcommand{\subparagraph}[1]{\oldsubparagraph{#1}\mbox{}}
\fi

%%% Use protect on footnotes to avoid problems with footnotes in titles
\let\rmarkdownfootnote\footnote%
\def\footnote{\protect\rmarkdownfootnote}

%%% Change title format to be more compact
\usepackage{titling}

% Create subtitle command for use in maketitle
\newcommand{\subtitle}[1]{
  \posttitle{
    \begin{center}\large#1\end{center}
    }
}

\setlength{\droptitle}{-2em}

  \title{R Notebook}
    \pretitle{\vspace{\droptitle}\centering\huge}
  \posttitle{\par}
    \author{}
    \preauthor{}\postauthor{}
    \date{}
    \predate{}\postdate{}
  

\begin{document}
\maketitle

\subsection{Linear model}\label{linear-model}

In this section, I take a look at hierarchical linear model of task
performance and try to predict task behavior using only a hierarchical
linear model.

This is a sensible baseline against which to measure the performance of
more complex hierarchical and joint models.

For hierarchical linear models of increasing complexity, standard
parametric estimation techniques can fail to produce accurate estimates
(Gelman, 2005). In these cases, it is advantageous to use a Bayesian
approach to estimation. The rstanarm package allows for the estimation
of a linear model using the Bayesian MCMC approach built into the stan
package.

\subsubsection{Method}\label{method}

I thus specified a hierarchical logistic regression model to test the
ability to predict correct vs.~incorrect responses \(C\) from both fixed
and mixed effects predictors. The fixed effects I specified are reaction
time \(RT\), Motivation (reward vs.~punishment) \(M\), meth use
\(METH\), sex risk \(SR\), and the position \(P\) of a presentation in a
pre- or post- reversal segment of cue presentations. I also specified
individual intercepts and slopes on the presentation position for each
subject \(S\) and for each individual run \(R\). I also specified a
separate mixed effect intercept on each image to represent ease of
recognizing any particular image cue \(CUE\).

\section{next section}\label{next-section}

This can be written as

\[\begin{aligned}
\text{logit}(C_{0 j}) & = \gamma_{0 0}+\gamma_{0 1} M_{ij}+\gamma_{0 2} RT_{ij}+\gamma_{0 3} RTS_{ij}+\gamma_{0 4} METH_{ij}+\gamma_{0 5} SR_{ij}+\gamma_{0 6} P_{ij} + e_{i j} \\
S_{1 j} & = \gamma_{10} + \gamma_{1 1} P_{1j} + R_{2j}\\
CUE_{2 j} & = \gamma_{10} + u_{2j}
\end{aligned}\]

The R code is written as

\begin{Shaded}
\begin{Highlighting}[]
\NormalTok{correct ~}\StringTok{ }\NormalTok{Motivation +}\StringTok{ }\NormalTok{reaction_time +}\StringTok{ }\NormalTok{(reaction_time <}\StringTok{ }\FloatTok{0.1}\NormalTok{) +}\StringTok{ }\NormalTok{presentation_n_in_segment +}\StringTok{ }
\StringTok{    }\NormalTok{MethUse +}\StringTok{ }\NormalTok{SexRisk +}\StringTok{ }\NormalTok{(}\DecValTok{1} \NormalTok{+}\StringTok{ }\NormalTok{presentation_n_in_segment |}\StringTok{ }\NormalTok{subid/runmotiveid) +}\StringTok{ }
\StringTok{    }\NormalTok{(}\DecValTok{1} \NormalTok{|}\StringTok{ }\NormalTok{image)}
\end{Highlighting}
\end{Shaded}

Reaction times recorded at less than 0.1 were considered as a separate
regressor because we assume that a genuine reaction time of less than
0.1 is not possible; thus those recorded at less than 0.1 must be button
presses unrelated to the actual stimulus, or perhaps a subject
anticipating the appearance of a stimulus and responding to the
anticipation rather than the stimulus itself. The specific cue presented
could not be anticipated, and thus subjects could not actually perform
better than chance, but they may be able to anticipate the appearance of
\emph{some} cue and respond to that.

\subsubsection{Result}\label{result}

\paragraph{Reliability}\label{reliability}

The effective sample size and \(\hat{R}\) values for all parameters were
within generally within the desired ranges, with the exception of the
Motivation parameter, which appeared to show good reliability from its
\(\hat{R}\) value but may have had poor efficiency.

\begin{longtable}[]{@{}lrrrrrrrr@{}}
\toprule
& mean & mcse & sd & 2.5\% & 50\% & 97.5\% & n\_eff &
Rhat\tabularnewline
\midrule
\endhead
(Intercept) & -0.99 & 0 & 0.05 & -1.10 & -0.99 & -0.89 & 1315 &
1.00\tabularnewline
Motivationreward & 0.02 & 0 & 0.05 & -0.07 & 0.02 & 0.11 & 575 &
1.01\tabularnewline
reaction time & 0.61 & 0 & 0.04 & 0.53 & 0.61 & 0.70 & 4000 &
1.00\tabularnewline
reaction time \textless{} 0.1TRUE & 0.22 & 0 & 0.16 & -0.09 & 0.22 &
0.53 & 4000 & 1.00\tabularnewline
presentation n in segment & 0.32 & 0 & 0.02 & 0.29 & 0.32 & 0.36 & 1078
& 1.00\tabularnewline
MethUseTRUE & -0.01 & 0 & 0.03 & -0.07 & -0.01 & 0.06 & 2825 &
1.00\tabularnewline
SexRiskTRUE & 0.02 & 0 & 0.03 & -0.04 & 0.02 & 0.08 & 2547 &
1.00\tabularnewline
\bottomrule
\end{longtable}

\paragraph{Fixed effects}\label{fixed-effects}

Fixed effects from the regression are shown below.

Of the parameters, reaction time and position in segment showed strong
predictive effects. We weren't able to detect predictive effects of
motivation type, meth use, or sex risk in this model.

\subsection{Discussion}\label{discussion}

Although the logistic regression model predicting correct response did
not show clear group differences, we can see within-run effects in
expected and interesting directions. Subjects with slower reaction times
tend to be more likely to get a correct response (95\% CI
\(b_{RT}=[0.53, 0.70]\)), suggesting that subjects who respond quickly
may be too hasty. With each presentation of a stimulus, within the
pre-reversal or post-reversal segments, subjects improve their
performance, with a roughly {[}7\%, 9\%{]} performance increase with
each trial.


\end{document}
