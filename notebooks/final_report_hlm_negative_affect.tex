\documentclass[]{article}
\usepackage{lmodern}
\usepackage{amssymb,amsmath}
\usepackage{ifxetex,ifluatex}
\usepackage{fixltx2e} % provides \textsubscript
\ifnum 0\ifxetex 1\fi\ifluatex 1\fi=0 % if pdftex
  \usepackage[T1]{fontenc}
  \usepackage[utf8]{inputenc}
\else % if luatex or xelatex
  \ifxetex
    \usepackage{mathspec}
  \else
    \usepackage{fontspec}
  \fi
  \defaultfontfeatures{Ligatures=TeX,Scale=MatchLowercase}
\fi
% use upquote if available, for straight quotes in verbatim environments
\IfFileExists{upquote.sty}{\usepackage{upquote}}{}
% use microtype if available
\IfFileExists{microtype.sty}{%
\usepackage{microtype}
\UseMicrotypeSet[protrusion]{basicmath} % disable protrusion for tt fonts
}{}
\usepackage[margin=1in]{geometry}
\usepackage{hyperref}
\hypersetup{unicode=true,
            pdftitle={Pain signal correlates with activity},
            pdfborder={0 0 0},
            breaklinks=true}
\urlstyle{same}  % don't use monospace font for urls
\usepackage{longtable,booktabs}
\usepackage{graphicx,grffile}
\makeatletter
\def\maxwidth{\ifdim\Gin@nat@width>\linewidth\linewidth\else\Gin@nat@width\fi}
\def\maxheight{\ifdim\Gin@nat@height>\textheight\textheight\else\Gin@nat@height\fi}
\makeatother
% Scale images if necessary, so that they will not overflow the page
% margins by default, and it is still possible to overwrite the defaults
% using explicit options in \includegraphics[width, height, ...]{}
\setkeys{Gin}{width=\maxwidth,height=\maxheight,keepaspectratio}
\IfFileExists{parskip.sty}{%
\usepackage{parskip}
}{% else
\setlength{\parindent}{0pt}
\setlength{\parskip}{6pt plus 2pt minus 1pt}
}
\setlength{\emergencystretch}{3em}  % prevent overfull lines
\providecommand{\tightlist}{%
  \setlength{\itemsep}{0pt}\setlength{\parskip}{0pt}}
\setcounter{secnumdepth}{0}
% Redefines (sub)paragraphs to behave more like sections
\ifx\paragraph\undefined\else
\let\oldparagraph\paragraph
\renewcommand{\paragraph}[1]{\oldparagraph{#1}\mbox{}}
\fi
\ifx\subparagraph\undefined\else
\let\oldsubparagraph\subparagraph
\renewcommand{\subparagraph}[1]{\oldsubparagraph{#1}\mbox{}}
\fi

%%% Use protect on footnotes to avoid problems with footnotes in titles
\let\rmarkdownfootnote\footnote%
\def\footnote{\protect\rmarkdownfootnote}

%%% Change title format to be more compact
\usepackage{titling}

% Create subtitle command for use in maketitle
\newcommand{\subtitle}[1]{
  \posttitle{
    \begin{center}\large#1\end{center}
    }
}

\setlength{\droptitle}{-2em}

  \title{Pain signal correlates with activity}
    \pretitle{\vspace{\droptitle}\centering\huge}
  \posttitle{\par}
    \author{}
    \preauthor{}\postauthor{}
    \date{}
    \predate{}\postdate{}
  

\begin{document}
\maketitle

\section{Pain signature correlates with behavioral performance in the
punishment
task}\label{pain-signature-correlates-with-behavioral-performance-in-the-punishment-task}

In the literature, `negative reinforcer' typically refers to an aversive
stimulus that is removed when a correct response is performed; `positive
punishment' typically refers to an aversive stimulus that is applied
when an incorrect response is performed, while `negative punishment'
refers to an reinforcer that is removed when an incorrect response is
performed. `Negative punishment' might include taking away money a
subject has been given following an incorrect response, while `negative
reinforcment might include' continually applying an unpleasant sensation
or loud noise until In our dataset, subjects received positive
reinforcement when they made the correct response to a trial in the
positive condition, and received an unconditioned positive punisher in
the form of an electric shock to the wrist following an incorrect
response to a trial in the punishment condition.

As far as I am aware there is no published work examining the effect of
a unconditioned primary positive pain punisher on shaping human behavior
in
fMRI\footnote{Loud noises have certainly been used in fMRI studies as an aversive stimulus, but I am unaware of whether they have been used in a behavioral task to shape behavior through positive punishment. Their status as an unconditioned aversive stimulus is also unclear, whereas an electric shock is indisputably an unconditioned aversive stimulus}.
This is interesting because understanding this can help us to understand
the differing neural circuitries involved in positive pain punishment
compared to positive reinforcement. An fMRI punishment approach is
particularly helpful because it helps us to understand the differences
in a controlled experimental setting.

In our joint modeling approach, a natural way to examine the differences
is to examine differences in parameters for positive compared to
negative reinforcement runs. We may examine model parameters such as the
learning rate, as well as other parameters such as the covariance of
expected value and reward prediction error with punishment cues.

\subsubsection{Neural pain signature}\label{neural-pain-signature}

The Neural Pain Signature (NPS) is a brain image from Tor Wager's 2013
paper on brain activity correlated with pain intensity. Rather than
attempting to identify clusters, Wager (2013) identified a whole-brain
measure, examining the extent to which each individual voxel correlates
with pain intensity. In doing so, he was able to get a very fine-grained
brain image that expresses in standard space the extent to which each
voxel correlates with pain intensity.

\subsubsection{NPS in this task}\label{nps-in-this-task}

If the NPS signature is generalizable to our subjects, then we should
see that when subjects were in the punishment condition, their NPS
activity should be elevated when they receive positive punishment. We
can predict pain using a hierarchical linear model, in which responding
correctly to a trial and punishment condition are included as
independent variables. A main effect of punishment condition would
suggest an extended pain feeling associated with the condition. This
might be variously interpreted as residual physical pain felt after the
electric shock and carrying over into subsequent trials, or as
anticipatory pain or anxiety. A main effect of incorrect response would
be interesting and suggest that the pain intensity signal identified by
Wager et al (2013) can also capture what might be interpreted as
`non-physical pain' associated with failing to respond correctly. An
interaction effect would suggest that our subjects really felt pain when
experiencing the aversive stimulus, and that this was not associated
with disappointment from failing to respond correctly. Any of the above
would suggest that the NPS is generalizable to our subjects in some
form. A failure to detect an effect could variously suggest that the NPS
isn't generalizable to our subjects, or that our electric shock was not
substantially painful, or could suggest a simple analysis error.

\subsubsection{Predictions}\label{predictions}

If the NPS similarity reliably measures pain, then we should see that
the NPS score is higher in trials where subjects receive an electric
shock compared to trials where they do not. Thus, sign that the NPS is
measuring pain is to ensure that:

\begin{enumerate}
\def\labelenumi{\arabic{enumi}.}
\tightlist
\item
  In the punishment runs, NPS scores should be higher in those trials
  where subjects receive an electric shock than those where they did
  not, i.e., where they made an incorrect response or made no response
  compared to when they made a correct response.
\item
  The difference between incorrect or non-responses and correct
  responses should be substantially stronger in the punishments than in
  the reward runs, where subjects were being rewarded for correct
  responses, but not punished for incorrect responses. If this is indeed
  the case, then we have evidence the NPS is measuring the brain
  response to physical pain. If there is no difference, then the NPS
  might be detecting non-physical pain, or a negative signal which is
  not physical pain.
\end{enumerate}

\subsection{Method}\label{method}

After the NPS scores were obtained (see Section DATA PROCESSING
CHAPTER), I tested these two postulates using a series of linear models.

607 runs across 161 subjects in the reversal learning dataset were added
into a hierarchical linear model. For reasons discussed in
Section\textasciitilde{}REFERENCE THE OTHER LINEAR MODEL, I used
rstanarm to do the analysis.

I first built a base level model, for comparison, that attempts to
predict the NPS signal on a given trial without reference to whether or
not that trial was correct.

We represent subject as \(S\), run as \(S\), and image as \(\iota\);
then

\(V=\mathit{PRESENTATION}_{i} + S_{s(i)} + R_{r(s,i)} + \iota_{j(i)} + \epsilon\)\label{eq:PainBaseNoMotivation}

We can also run the model with Motivation, but not response:

\(V=M_k + S_{s(i)} + R_{r(s,i)} + \iota_{j(i)} + \epsilon\)\label{eq:PainBaseNoResponse}

Then we can repeat the model, this time with the correct response
included:

\(V=\mathit{RESPONSE}_k*M_k+\mathit{PRESENTATION}_{i} + S_{s(i, k )} + R_{r(s,i , k )} + \iota_{j(i, k)} + \epsilon\)
\footnote{I have to go back and see if this notation is really the notation I want to use! It's pretty wack, but I think it is fully expressing the different parts of the model properly. But this section should be consistent with the other section}

If the third model is significantly better than each of the first two,
then we have clear evidence that the interaction effect between response
and motivation condition (i.e., when subjects actually get an electric
shock) exists.

Neural pain signature values were mean centered for each subject across
all runs.

\subsection{Results}\label{results}

The first base pain model included information about response, but not
about motivation
(Equation\textasciitilde{}\ref{eq:PainBaseNoMotivation}). It can be
written in R like

\begin{verbatim}
ValueScaled~
  (ResponseCorrect==FALSE) + 
  presentation_n_in_segment + 
  (1+presentation_n_in_segment | subid/runmotiveid) + 
  (1 | image)
\end{verbatim}

The fixed effects of ReponseCorrect and presentation in segment are
shown below.

\begin{longtable}[]{@{}ccccc@{}}
\caption{Table continues below}\tabularnewline
\toprule
\begin{minipage}[b]{0.36\columnwidth}\centering\strut
~\strut
\end{minipage} & \begin{minipage}[b]{0.12\columnwidth}\centering\strut
mean\strut
\end{minipage} & \begin{minipage}[b]{0.13\columnwidth}\centering\strut
mcse\strut
\end{minipage} & \begin{minipage}[b]{0.12\columnwidth}\centering\strut
sd\strut
\end{minipage} & \begin{minipage}[b]{0.12\columnwidth}\centering\strut
2.5\%\strut
\end{minipage}\tabularnewline
\midrule
\endfirsthead
\toprule
\begin{minipage}[b]{0.36\columnwidth}\centering\strut
~\strut
\end{minipage} & \begin{minipage}[b]{0.12\columnwidth}\centering\strut
mean\strut
\end{minipage} & \begin{minipage}[b]{0.13\columnwidth}\centering\strut
mcse\strut
\end{minipage} & \begin{minipage}[b]{0.12\columnwidth}\centering\strut
sd\strut
\end{minipage} & \begin{minipage}[b]{0.12\columnwidth}\centering\strut
2.5\%\strut
\end{minipage}\tabularnewline
\midrule
\endhead
\begin{minipage}[t]{0.36\columnwidth}\centering\strut
\textbf{(Intercept)}\strut
\end{minipage} & \begin{minipage}[t]{0.12\columnwidth}\centering\strut
-0.05648\strut
\end{minipage} & \begin{minipage}[t]{0.13\columnwidth}\centering\strut
0.001117\strut
\end{minipage} & \begin{minipage}[t]{0.12\columnwidth}\centering\strut
0.03117\strut
\end{minipage} & \begin{minipage}[t]{0.12\columnwidth}\centering\strut
-0.1171\strut
\end{minipage}\tabularnewline
\begin{minipage}[t]{0.36\columnwidth}\centering\strut
\textbf{ResponseCorrect == FALSETRUE}\strut
\end{minipage} & \begin{minipage}[t]{0.12\columnwidth}\centering\strut
0.05476\strut
\end{minipage} & \begin{minipage}[t]{0.13\columnwidth}\centering\strut
9.255e-05\strut
\end{minipage} & \begin{minipage}[t]{0.12\columnwidth}\centering\strut
0.005854\strut
\end{minipage} & \begin{minipage}[t]{0.12\columnwidth}\centering\strut
0.04352\strut
\end{minipage}\tabularnewline
\begin{minipage}[t]{0.36\columnwidth}\centering\strut
\textbf{presentation\_n\_in\_segment}\strut
\end{minipage} & \begin{minipage}[t]{0.12\columnwidth}\centering\strut
0.01175\strut
\end{minipage} & \begin{minipage}[t]{0.13\columnwidth}\centering\strut
3.287e-05\strut
\end{minipage} & \begin{minipage}[t]{0.12\columnwidth}\centering\strut
0.002079\strut
\end{minipage} & \begin{minipage}[t]{0.12\columnwidth}\centering\strut
0.00774\strut
\end{minipage}\tabularnewline
\bottomrule
\end{longtable}

\begin{longtable}[]{@{}ccccc@{}}
\toprule
\begin{minipage}[b]{0.38\columnwidth}\centering\strut
~\strut
\end{minipage} & \begin{minipage}[b]{0.13\columnwidth}\centering\strut
50\%\strut
\end{minipage} & \begin{minipage}[b]{0.13\columnwidth}\centering\strut
97.5\%\strut
\end{minipage} & \begin{minipage}[b]{0.10\columnwidth}\centering\strut
n\_eff\strut
\end{minipage} & \begin{minipage}[b]{0.10\columnwidth}\centering\strut
Rhat\strut
\end{minipage}\tabularnewline
\midrule
\endhead
\begin{minipage}[t]{0.38\columnwidth}\centering\strut
\textbf{(Intercept)}\strut
\end{minipage} & \begin{minipage}[t]{0.13\columnwidth}\centering\strut
-0.05703\strut
\end{minipage} & \begin{minipage}[t]{0.13\columnwidth}\centering\strut
0.003106\strut
\end{minipage} & \begin{minipage}[t]{0.10\columnwidth}\centering\strut
778\strut
\end{minipage} & \begin{minipage}[t]{0.10\columnwidth}\centering\strut
1.004\strut
\end{minipage}\tabularnewline
\begin{minipage}[t]{0.38\columnwidth}\centering\strut
\textbf{ResponseCorrect == FALSETRUE}\strut
\end{minipage} & \begin{minipage}[t]{0.13\columnwidth}\centering\strut
0.05467\strut
\end{minipage} & \begin{minipage}[t]{0.13\columnwidth}\centering\strut
0.06628\strut
\end{minipage} & \begin{minipage}[t]{0.10\columnwidth}\centering\strut
4000\strut
\end{minipage} & \begin{minipage}[t]{0.10\columnwidth}\centering\strut
0.9998\strut
\end{minipage}\tabularnewline
\begin{minipage}[t]{0.38\columnwidth}\centering\strut
\textbf{presentation\_n\_in\_segment}\strut
\end{minipage} & \begin{minipage}[t]{0.13\columnwidth}\centering\strut
0.01173\strut
\end{minipage} & \begin{minipage}[t]{0.13\columnwidth}\centering\strut
0.01588\strut
\end{minipage} & \begin{minipage}[t]{0.10\columnwidth}\centering\strut
4000\strut
\end{minipage} & \begin{minipage}[t]{0.10\columnwidth}\centering\strut
0.9991\strut
\end{minipage}\tabularnewline
\bottomrule
\end{longtable}

Both Response Incorrect and Presentation N postively predict pain.

I then build a second comparison model. This one includes information
about motives, but not whether the response in each trial is correct.

\begin{longtable}[]{@{}lrrrrrrrr@{}}
\toprule
& mean & mcse & sd & 2.5\% & 50\% & 97.5\% & n\_eff &
Rhat\tabularnewline
\midrule
\endhead
(Intercept) & -0.0564802 & 0.0011174 & 0.0311705 & -0.1171299 &
-0.0570284 & 0.0031065 & 778 & 1.0035026\tabularnewline
ResponseCorrect == FALSETRUE & 0.0547575 & 0.0000926 & 0.0058535 &
0.0435186 & 0.0546743 & 0.0662753 & 4000 & 0.9998315\tabularnewline
presentation\_n\_in\_segment & 0.0117493 & 0.0000329 & 0.0020791 &
0.0077404 & 0.0117348 & 0.0158829 & 4000 & 0.9991234\tabularnewline
\bottomrule
\end{longtable}

Finally, we can examine the model that includes both Motivation and
Trial Response:

\begin{verbatim}
## 
## 
##                                                                      mean        mcse          sd         2.5%          50%        97.5%   n_eff        Rhat
## ------------------------------------------------------------  -----------  ----------  ----------  -----------  -----------  -----------  ------  ----------
## (Intercept)                                                    -0.0174258   0.0013238   0.0344792   -0.0843168   -0.0176780    0.0491088     678   1.0022331
## ResponseCorrect == FALSETRUE                                   -0.0020475   0.0001358   0.0085867   -0.0186858   -0.0021165    0.0146291    4000   0.9992762
## Motivation == "punishment"TRUE                                 -0.0747125   0.0006274   0.0238127   -0.1233088   -0.0742514   -0.0287189    1440   1.0023112
## presentation_n_in_segment                                       0.0114784   0.0000321   0.0020278    0.0074069    0.0115253    0.0154053    4000   0.9998858
## ResponseCorrect == FALSETRUE:Motivation == "punishment"TRUE     0.1018704   0.0001804   0.0114082    0.0790453    0.1019392    0.1244361    4000   0.9994433
\end{verbatim}

In this model, there was a main effect of Punishment, indicating that
even when controlling for the specific experience of pain during the
incorrect trials in the punishment condition, there was evidence overall
of slightly more pain experience in the Punishment condition than in the
Reward condition.

\subsection{Model comparison}\label{model-comparison}

Because these models are Bayesian models, standard AIC and BIC tests are
less useful. However we can apply Pareto smoothed importance-sampling
leave-one-out cross-validation
\footnote{(Vehtari, A., Gelman, A., and Gabry, J. (2017). Practical Bayesian model evaluation using leave-one-out cross-validation and WAIC. Statistics and Computing. 27(5), 1413–1432. :10.1007/s11222-016-9696-4. Links: published | arXiv preprint.

Vehtari, A., Gelman, A., and Gabry, J. (2017). Pareto smoothed importance sampling. arXiv preprint arXiv:1507.04544.)}

\begin{verbatim}
## This is loo version 2.0.0.
## **NOTE: As of version 2.0.0 loo defaults to 1 core but we recommend using as many as possible. Use the 'cores' argument or set options(mc.cores = NUM_CORES) for an entire session. Visit mc-stan.org/loo/news for details on other changes.
\end{verbatim}

BIC indicates that the model including punishment is strongly
predictive.

However the clearest indication of the result here, since we are
interested in the direction of an effect, will come from examining the
significance and magnitude of the effect sizes of the regressors
themselves:

The result:

\begin{enumerate}
\def\labelenumi{\arabic{enumi}.}
\tightlist
\item
  With an estimate of \(b=0.102\) (CI=\([0.079,0.124]\)), the
  interaction of Incorrect Response and Motivation is strong and
  indicates that there was a brain response to an incorrect outcome in
  the electric shock punishment condition but not in the reward
  condition. The interaction shows up particularly clear in the subject
  aggregates in table XX which show a large difference in NPS activity
  between incorrect and correct trials in the Punishment condition but
  not in the Reward condition.
\item
  The main effect of motivation actually turns out to be in the reverse
  of the predicted direction: the magnitude of the NPS signal is
  \emph{lower} for the Pain condition than the Reward condition
  (\(b=-0.075\), Ci=\([-0.074, -0.029]\)). It should be remembered that
  when we combine this with the interaction term,
  \(b_{Motiv=punish} + b_{Motiv=Punish*Response=Incorrect}\), we see
  that the predicted deviation from baseline is
  \(0.102\times 0 - 0.075=-0.075\) during the Punishment condition when
  there is no Punishment signal (i.e., response is correct), but
  \(0.102\times 1 - 0.075=0.027\) during the Punishment condition in
  trials that do contain a Punishment signal. This effect may arise from
  normalization of the NPS signal of the Reward and Punishment
  conditions separately.
\end{enumerate}

The graph below shows subjects' mean pain responses in the reward and
punishment conditions, when they got a response correct or a response
incorrect. Subjects who did not have at least one run in each of the
motivation conditions were excluded. Values

\includegraphics{final_report_hlm_negative_affect_files/figure-latex/finalgraphs2-1.pdf}

\includegraphics{final_report_hlm_negative_affect_files/figure-latex/finalgraphs3-1.pdf}

\includegraphics{final_report_hlm_negative_affect_files/figure-latex/finalgraphs4-1.pdf}

\section{Pain signal section
discussion}\label{pain-signal-section-discussion}

This may be the first time that a generalized neurologic pain signal has
been connected to an electric shock stimulus.

There are problems with the model used as it doesn't fully account for
all the levels of variance. To do that we need to use a Bayesian model.

I ran a Bayesian model using rstanarm. In this design, I included random
effects of correct response and presentation order, as well as intercept
and presentation order within segment for each subject and each run
within each subject. This was more variance than I could account for in
a straight linear model.

The 95\% HDI for the beta value for ResponseCorrect\(\times\)Punishment
interaction was {[}-0.11, -0.04{]}, reflecting that high probability
that an incorrect response would lead to a pain signal in the Punishment
condition only.

The graphical data is puzzling, and suggests that although there was a
difference in NPS visible within punishment condition between correct
and incorrect responses, the level of pain observed during the
punishment condition was comparable to the level of pain experienced in
the reward condition. This may be an artefact of the analysis process:
becasue they were separate runs, each were overall mean-centered
separately.

Overall, the result confirms the initial hypothesis. There was a
substantial difference between wrong and correct trials in the
punishment condition. There was no evidence for residual pain in the
punishment condition: correct responses in the punishment condition had,
if anything, lower pain responses than correct responses in the reward
condition (Figure X). There was also no evidence for a difference in
experienced pain between correct and incorrect responses in the reward
condition (Figure Y), suggesting that only physical pain through a
positive punisher triggered the NPS response while negative
reinforcement (losing the opportunity for a reward after an incorrect
response in the reward condition) did not yield the same result.


\end{document}
