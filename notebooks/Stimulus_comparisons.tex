\documentclass[]{article}
\usepackage{lmodern}
\usepackage{amssymb,amsmath}
\usepackage{ifxetex,ifluatex}
\usepackage{fixltx2e} % provides \textsubscript
\ifnum 0\ifxetex 1\fi\ifluatex 1\fi=0 % if pdftex
  \usepackage[T1]{fontenc}
  \usepackage[utf8]{inputenc}
\else % if luatex or xelatex
  \ifxetex
    \usepackage{mathspec}
  \else
    \usepackage{fontspec}
  \fi
  \defaultfontfeatures{Ligatures=TeX,Scale=MatchLowercase}
\fi
% use upquote if available, for straight quotes in verbatim environments
\IfFileExists{upquote.sty}{\usepackage{upquote}}{}
% use microtype if available
\IfFileExists{microtype.sty}{%
\usepackage{microtype}
\UseMicrotypeSet[protrusion]{basicmath} % disable protrusion for tt fonts
}{}
\usepackage[margin=1in]{geometry}
\usepackage{hyperref}
\hypersetup{unicode=true,
            pdftitle={Stimulus comparisons},
            pdfborder={0 0 0},
            breaklinks=true}
\urlstyle{same}  % don't use monospace font for urls
\usepackage{longtable,booktabs}
\usepackage{graphicx,grffile}
\makeatletter
\def\maxwidth{\ifdim\Gin@nat@width>\linewidth\linewidth\else\Gin@nat@width\fi}
\def\maxheight{\ifdim\Gin@nat@height>\textheight\textheight\else\Gin@nat@height\fi}
\makeatother
% Scale images if necessary, so that they will not overflow the page
% margins by default, and it is still possible to overwrite the defaults
% using explicit options in \includegraphics[width, height, ...]{}
\setkeys{Gin}{width=\maxwidth,height=\maxheight,keepaspectratio}
\IfFileExists{parskip.sty}{%
\usepackage{parskip}
}{% else
\setlength{\parindent}{0pt}
\setlength{\parskip}{6pt plus 2pt minus 1pt}
}
\setlength{\emergencystretch}{3em}  % prevent overfull lines
\providecommand{\tightlist}{%
  \setlength{\itemsep}{0pt}\setlength{\parskip}{0pt}}
\setcounter{secnumdepth}{0}
% Redefines (sub)paragraphs to behave more like sections
\ifx\paragraph\undefined\else
\let\oldparagraph\paragraph
\renewcommand{\paragraph}[1]{\oldparagraph{#1}\mbox{}}
\fi
\ifx\subparagraph\undefined\else
\let\oldsubparagraph\subparagraph
\renewcommand{\subparagraph}[1]{\oldsubparagraph{#1}\mbox{}}
\fi

%%% Use protect on footnotes to avoid problems with footnotes in titles
\let\rmarkdownfootnote\footnote%
\def\footnote{\protect\rmarkdownfootnote}

%%% Change title format to be more compact
\usepackage{titling}

% Create subtitle command for use in maketitle
\newcommand{\subtitle}[1]{
  \posttitle{
    \begin{center}\large#1\end{center}
    }
}

\setlength{\droptitle}{-2em}

  \title{Stimulus comparisons}
    \pretitle{\vspace{\droptitle}\centering\huge}
  \posttitle{\par}
    \author{}
    \preauthor{}\postauthor{}
    \date{}
    \predate{}\postdate{}
  

\begin{document}
\maketitle

How many stimuli would we need to compare to get a score of
`confusability' between images?

All the images are presented in sequence, and they are the same sequence
for each subject:

\begin{longtable}[]{@{}lrlll@{}}
\toprule
Motivation & runid & subid 148 & subid 249 & subid 350\tabularnewline
\midrule
\endhead
punishment & 1 & 51, 52, 53, 54, 55, 56, 57, 58, 59, 60, 61, 62, 63, 64,
65, 66, 67, 68, 69, 70, 71, 72 & 51, 52, 53, 54, 55, 56, 57, 58, 59, 60,
61, 62, 63, 64, 65, 66, 67, 68, 69, 70, 71, 72 & 51, 52, 53, 54, 55, 56,
57, 58, 59, 60, 61, 62, 63, 64, 65, 66, 67, 68, 69, 70, 71,
72\tabularnewline
punishment & 2 & 75, 76, 77, 78, 79, 80, 81, 82, 83, 84, 85, 86, 87, 88,
89, 90, 91, 92, 93, 94, 95, 96 & 75, 76, 77, 78, 79, 80, 81, 82, 83, 84,
85, 86, 87, 88, 89, 90, 91, 92, 93, 94, 95, 96 & 75, 76, 77, 78, 79, 80,
81, 82, 83, 84, 85, 86, 87, 88, 89, 90, 91, 92, 93, 94, 95,
96\tabularnewline
reward & 1 & 1, 2, 3, 4, 5, 6, 7, 8, 9, 10, 11, 12, 13, 14, 15, 16, 17,
18, 19, 20, 21, 22 & 1, 2, 3, 4, 5, 6, 7, 8, 9, 10, 11, 12, 13, 14, 15,
16, 17, 18, 19, 20, 21, 22 & 1, 2, 3, 4, 5, 6, 7, 8, 9, 10, 11, 12, 13,
14, 15, 16, 17, 18, 19, 20, 21, 22\tabularnewline
reward & 2 & 25, 26, 27, 28, 29, 30, 31, 32, 33, 34, 35, 36, 37, 38, 39,
40, 41, 42, 43, 44, 45, 46 & 25, 26, 27, 28, 29, 30, 31, 32, 33, 34, 35,
36, 37, 38, 39, 40, 41, 42, 43, 44, 45, 46 & 25, 26, 27, 28, 29, 30, 31,
32, 33, 34, 35, 36, 37, 38, 39, 40, 41, 42, 43, 44, 45,
46\tabularnewline
\bottomrule
\end{longtable}

The simplest way to get a measure of ``confusability'' here might be to
present exactly the same sequence of images and ask subject, ``Have you
seen this image before?'' If subjects say ``yes'' when they have not in
fact seen the image before, we could infer that the images are more
``confusable''. In this design, we wouldn't know which image is being
confused.

If we do some 2x2 test, we might want to comapre all images with all the
images that preceded them. We can calculate a matrix describing how many
timesteps prior to each image each other image was observed. We should
do this separately for run1, run2, reward, and punishment; it only needs
to be done with one subject. We can pick subject 229 as one subject we
know has all data intact.

\begin{verbatim}
## 
## punishment     reward 
##        436        436
\end{verbatim}

\begin{verbatim}
## 
##   1   2 
## 436 436
\end{verbatim}

The total number of comparisons is 348. If we had to run each comparison
only once per subject, and each trial took 3 seconds, the length of the
task would be 17.4 minutes. With 270 comparisons, this drops to 13.5
minutes.


\end{document}
