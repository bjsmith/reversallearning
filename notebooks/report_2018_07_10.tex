\documentclass[]{article}
\usepackage{lmodern}
\usepackage{amssymb,amsmath}
\usepackage{ifxetex,ifluatex}
\usepackage{fixltx2e} % provides \textsubscript
\ifnum 0\ifxetex 1\fi\ifluatex 1\fi=0 % if pdftex
  \usepackage[T1]{fontenc}
  \usepackage[utf8]{inputenc}
\else % if luatex or xelatex
  \ifxetex
    \usepackage{mathspec}
  \else
    \usepackage{fontspec}
  \fi
  \defaultfontfeatures{Ligatures=TeX,Scale=MatchLowercase}
\fi
% use upquote if available, for straight quotes in verbatim environments
\IfFileExists{upquote.sty}{\usepackage{upquote}}{}
% use microtype if available
\IfFileExists{microtype.sty}{%
\usepackage{microtype}
\UseMicrotypeSet[protrusion]{basicmath} % disable protrusion for tt fonts
}{}
\usepackage[margin=1in]{geometry}
\usepackage{hyperref}
\hypersetup{unicode=true,
            pdftitle={Report-2018-07-10},
            pdfauthor={Ben Smith},
            pdfborder={0 0 0},
            breaklinks=true}
\urlstyle{same}  % don't use monospace font for urls
\usepackage{longtable,booktabs}
\usepackage{graphicx,grffile}
\makeatletter
\def\maxwidth{\ifdim\Gin@nat@width>\linewidth\linewidth\else\Gin@nat@width\fi}
\def\maxheight{\ifdim\Gin@nat@height>\textheight\textheight\else\Gin@nat@height\fi}
\makeatother
% Scale images if necessary, so that they will not overflow the page
% margins by default, and it is still possible to overwrite the defaults
% using explicit options in \includegraphics[width, height, ...]{}
\setkeys{Gin}{width=\maxwidth,height=\maxheight,keepaspectratio}
\IfFileExists{parskip.sty}{%
\usepackage{parskip}
}{% else
\setlength{\parindent}{0pt}
\setlength{\parskip}{6pt plus 2pt minus 1pt}
}
\setlength{\emergencystretch}{3em}  % prevent overfull lines
\providecommand{\tightlist}{%
  \setlength{\itemsep}{0pt}\setlength{\parskip}{0pt}}
\setcounter{secnumdepth}{0}
% Redefines (sub)paragraphs to behave more like sections
\ifx\paragraph\undefined\else
\let\oldparagraph\paragraph
\renewcommand{\paragraph}[1]{\oldparagraph{#1}\mbox{}}
\fi
\ifx\subparagraph\undefined\else
\let\oldsubparagraph\subparagraph
\renewcommand{\subparagraph}[1]{\oldsubparagraph{#1}\mbox{}}
\fi

%%% Use protect on footnotes to avoid problems with footnotes in titles
\let\rmarkdownfootnote\footnote%
\def\footnote{\protect\rmarkdownfootnote}

%%% Change title format to be more compact
\usepackage{titling}

% Create subtitle command for use in maketitle
\newcommand{\subtitle}[1]{
  \posttitle{
    \begin{center}\large#1\end{center}
    }
}

\setlength{\droptitle}{-2em}

  \title{Report-2018-07-10}
    \pretitle{\vspace{\droptitle}\centering\huge}
  \posttitle{\par}
    \author{Ben Smith}
    \preauthor{\centering\large\emph}
  \postauthor{\par}
      \predate{\centering\large\emph}
  \postdate{\par}
    \date{7/10/2018}


\begin{document}
\maketitle

\section{Reward learning EV model}\label{reward-learning-ev-model}

I tested a set of regions defined by freesurfer for their covariance
with expected value and reward prediction error.

In this analysis, about 3/4 of the runs were included, but some runs
were excluded because stan was unable to converge on their results. I'm
working on a re-analysis now that should include more runs.

\subsection{Reward prediction error}\label{reward-prediction-error}

We can see how activity in 37 regions representing parts of the striatum
and ventromedial prefrontal cortex correlate with reward prediction
error:

\begin{longtable}[]{@{}lll@{}}
\toprule
Region & RPECI95Pct & RPE\_FDRadjustedPValue\tabularnewline
\midrule
\endhead
Putamen (Left) & {[}0.04, 0.055{]} & \textless{} 2e-16\tabularnewline
Putamen (Right) & {[}0.039, 0.054{]} & \textless{} 2e-16\tabularnewline
Accumbens area (Left) & {[}0.027, 0.041{]} & \textless{}
2e-16\tabularnewline
Accumbens area (Right) & {[}0.025, 0.038{]} & 9.6e-16\tabularnewline
cingul Ant gyrus and sulcus (Left) & {[}0.02, 0.032{]} &
1.4e-14\tabularnewline
circular insula sup sulcus (Left) & {[}-0.037, -0.023{]} &
1.1e-13\tabularnewline
suborbital sulcus (Left) & {[}0.018, 0.032{]} & 3.6e-10\tabularnewline
cingul Ant gyrus and sulcus (Right) & {[}0.016, 0.029{]} &
1.3e-09\tabularnewline
cingul Post dorsal gyrus (Right) & {[}0.02, 0.036{]} &
1.6e-09\tabularnewline
subcallosal gyrus (Left) & {[}0.013, 0.025{]} & 4.6e-09\tabularnewline
circular insula sup sulcus (Right) & {[}-0.03, -0.016{]} &
9.4e-09\tabularnewline
front middle sulcus (Left) & {[}-0.026, -0.014{]} &
4.0e-08\tabularnewline
Caudate (Left) & {[}0.013, 0.025{]} & 5.5e-08\tabularnewline
suborbital sulcus (Right) & {[}0.011, 0.024{]} & 4.4e-07\tabularnewline
cingul Mid Ant gyrus and sulcus (Left) & {[}-0.025, -0.011{]} &
1.5e-06\tabularnewline
Caudate (Right) & {[}0.009, 0.021{]} & 1.3e-05\tabularnewline
insular short gyrus (Right) & {[}-0.021, -0.0085{]} &
4.0e-05\tabularnewline
Pallidum (Left) & {[}0.0077, 0.02{]} & 7.3e-05\tabularnewline
Thalamus Proper (Right) & {[}-0.02, -0.0065{]} & 0.00042\tabularnewline
cingul Mid Ant gyrus and sulcus (Right) & {[}-0.02, -0.0064{]} &
0.00067\tabularnewline
front sup sulcus (Left) & {[}0.0059, 0.019{]} & 0.00073\tabularnewline
subcallosal gyrus (Right) & {[}0.0055, 0.018{]} & 0.00085\tabularnewline
front middle sulcus (Right) & {[}-0.019, -0.0053{]} &
0.00158\tabularnewline
circular insula ant sulcus (Left) & {[}-0.02, -0.0053{]} &
0.00166\tabularnewline
Amygdala (Left) & {[}0.004, 0.016{]} & 0.00302\tabularnewline
insular short gyrus (Left) & {[}-0.017, -0.0041{]} &
0.00344\tabularnewline
occipital ant sulcus (Right) & {[}0.0029, 0.017{]} &
0.01282\tabularnewline
circular insula ant sulcus (Right) & {[}-0.017, -0.0024{]} &
0.01841\tabularnewline
front sup sulcus (Right) & {[}0.002, 0.015{]} & 0.02070\tabularnewline
front sup gyrus (Right) & {[}-0.015, -0.0013{]} & 0.03338\tabularnewline
Pallidum (Right) & {[}-0.00044, 0.014{]} & 0.09955\tabularnewline
rectus gyrus (Left) & {[}-0.0007, 0.01{]} & 0.12900\tabularnewline
temporal sup sulcus (Left) & {[}-0.00096, 0.012{]} &
0.13471\tabularnewline
Thalamus Proper (Left) & {[}-0.011, 0.0013{]} & 0.16310\tabularnewline
Amygdala (Right) & {[}-0.0015, 0.01{]} & 0.18537\tabularnewline
rectus gyrus (Right) & {[}-0.0031, 0.01{]} & 0.36709\tabularnewline
front sup gyrus (Left) & {[}-0.0052, 0.0073{]} & 0.78287\tabularnewline
\bottomrule
\end{longtable}

Most of the regions tested were significant.

Unfortunately, this did include a `control region', the occipital
anterior sulcus.

\subsection{Expected value}\label{expected-value}

We also can see how activity in 37 regions representing parts of the
striatum and ventromedial prefrontal cortex expected value:

\begin{longtable}[]{@{}lllr@{}}
\toprule
& Region & EVCI95Pct & EV\_FDRadjustedPValue\tabularnewline
\midrule
\endhead
29 & front sup sulcus (Right) & {[}0.0063, 0.016{]} &
0.000059\tabularnewline
9 & cingul Post dorsal gyrus (Right) & {[}0.0075, 0.02{]} &
0.000073\tabularnewline
20 & cingul Mid Ant gyrus and sulcus (Right) & {[}-0.016, -0.0056{]} &
0.000220\tabularnewline
11 & circular insula sup sulcus (Right) & {[}-0.016, -0.0052{]} &
0.000670\tabularnewline
6 & circular insula sup sulcus (Left) & {[}-0.016, -0.005{]} &
0.000850\tabularnewline
24 & circular insula ant sulcus (Left) & {[}-0.015, -0.0043{]} &
0.001280\tabularnewline
15 & cingul Mid Ant gyrus and sulcus (Left) & {[}-0.014, -0.004{]} &
0.001430\tabularnewline
28 & circular insula ant sulcus (Right) & {[}-0.016, -0.0043{]} &
0.001640\tabularnewline
12 & front middle sulcus (Left) & {[}-0.015, -0.0033{]} &
0.005140\tabularnewline
26 & insular short gyrus (Left) & {[}-0.016, -0.0035{]} &
0.005850\tabularnewline
1 & Putamen (Left) & {[}0.0018, 0.014{]} & 0.022320\tabularnewline
17 & insular short gyrus (Right) & {[}-0.013, -0.0016{]} &
0.022610\tabularnewline
8 & cingul Ant gyrus and sulcus (Right) & {[}0.0012, 0.01{]} &
0.023280\tabularnewline
14 & suborbital sulcus (Right) & {[}0.0017, 0.014{]} &
0.023280\tabularnewline
27 & occipital ant sulcus (Right) & {[}-0.014, -0.0013{]} &
0.031920\tabularnewline
7 & suborbital sulcus (Left) & {[}0.00038, 0.011{]} &
0.057550\tabularnewline
30 & front sup gyrus (Right) & {[}-0.00017, 0.0091{]} &
0.092930\tabularnewline
2 & Putamen (Right) & {[}-0.00035, 0.011{]} & 0.099550\tabularnewline
32 & rectus gyrus (Left) & {[}-0.01, 0.00072{]} &
0.129000\tabularnewline
4 & Accumbens area (Right) & {[}-0.011, 0.001{]} &
0.143350\tabularnewline
5 & cingul Ant gyrus and sulcus (Left) & {[}-0.0093, 0.001{]} &
0.160270\tabularnewline
36 & rectus gyrus (Right) & {[}-0.0019, 0.0095{]} &
0.244940\tabularnewline
18 & Pallidum (Left) & {[}-0.0024, 0.0096{]} & 0.298040\tabularnewline
22 & subcallosal gyrus (Right) & {[}-0.0028, 0.0095{]} &
0.354550\tabularnewline
25 & Amygdala (Left) & {[}-0.0095, 0.0031{]} & 0.385890\tabularnewline
21 & front sup sulcus (Left) & {[}-0.0031, 0.0088{]} &
0.409920\tabularnewline
16 & Caudate (Right) & {[}-0.0066, 0.0028{]} & 0.501480\tabularnewline
37 & front sup gyrus (Left) & {[}-0.007, 0.003{]} &
0.506220\tabularnewline
3 & Accumbens area (Left) & {[}-0.0037, 0.0081{]} &
0.522770\tabularnewline
10 & subcallosal gyrus (Left) & {[}-0.0067, 0.0034{]} &
0.575500\tabularnewline
19 & Thalamus Proper (Right) & {[}-0.0072, 0.0038{]} &
0.595590\tabularnewline
31 & Pallidum (Right) & {[}-0.0052, 0.008{]} & 0.735760\tabularnewline
13 & Caudate (Left) & {[}-0.0047, 0.0068{]} & 0.766600\tabularnewline
34 & Thalamus Proper (Left) & {[}-0.0067, 0.0049{]} &
0.784550\tabularnewline
35 & Amygdala (Right) & {[}-0.0069, 0.0051{]} & 0.786480\tabularnewline
33 & temporal sup sulcus (Left) & {[}-0.0054, 0.0066{]} &
0.852940\tabularnewline
23 & front middle sulcus (Right) & {[}-0.0053, 0.0062{]} &
0.872330\tabularnewline
\bottomrule
\end{longtable}

Several regions showed significant \emph{negative} correlations.

Regions that showed significant positive correlations were:

\begin{itemize}
\tightlist
\item
  In the dorsal prefrontal cortex, the following areas:
\item
  Right Superior frontal sulcus (between the SFG and MFG)
\item
  Right frontal ACC
\item
  In the striatum, the following areas:
\item
  Left Putamen
\item
  Posterior cingulate gyrus (dorsal)
\end{itemize}

Additionally there, were a number of regions with significant
\emph{negative} correlatoins with expected value:

\begin{itemize}
\tightlist
\item
  In the dorsal prefrontal cortex,
\item
  Left and right middle ACC
\item
  Left middle frontal sulcus
\item
  The control region Occipital anterior sulcus
\item
  left and right Anterior and Superior-posterior insula
\end{itemize}

Notably, we \emph{did not} see postiive correlations between expected
value and the ventromedial prefrontal cortex. This does undermine the
results somewhat.

Factors that may lead to a failure to detect valuation in the vmPFC
could include:

\begin{itemize}
\tightlist
\item
  `Expected value' means the expected value of the presented cue given
  the subject's history with the presented cue. However, in this task,
  subjects often confuse the identity of various cues, and this is
  likely the largest source of error. Yet this is not modeled.
\item
  We are only able to get a very blunt measure of correlations in this
  method because the model is not hierarchical.
\item
  vmPFC tracks value, but it's unclear whether this means that higher
  value should mean more activity in ROIs within the ventromedial
  prefrontal cortex.
\end{itemize}

\section{Next steps}\label{next-steps}


\end{document}
